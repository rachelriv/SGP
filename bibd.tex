\documentclass[12pt]{article}

\usepackage{url}
\usepackage[margin=1in]{geometry}  % set the margins to 1in on all sides
\usepackage{graphicx}              % to include figures
\usepackage{amsmath}               % great math stuff
\usepackage{amsfonts}              % for blackboard bold, etc
\usepackage{amsthm}                % better theorem environments
\usepackage{amssymb,bm}
\theoremstyle{definition}
\newtheorem{thm}{Theorem}[section]
\newtheorem{mydef}{Definition}
\newtheorem{ex}[thm]{Example}
\newtheorem{lem}[thm]{Lemma}
\newtheorem{prop}[thm]{Proposition}
\newtheorem{cor}[thm]{Corollary}
\newtheorem{conj}[thm]{Conjecture}

\DeclareMathOperator{\id}{id}

\newcommand{\bd}[1]{\mathbf{#1}}  % for bolding symbols
\newcommand{\RR}{\mathbb{R}}      % for Real numbers
\newcommand{\ZZ}{\mathbb{Z}}      % for Integers
\newcommand{\col}[1]{\left[\begin{matrix} #1 \end{matrix} \right]}
\newcommand{\comb}[2]{\binom{#1^2 + #2^2}{#1+#2}}


\begin{document}


\nocite{*}

\title{Balanced Incomplete Block Designs}

\author{Rachel Rivera\\ 
Computer Science Department \\
Loyola Marymount University \\
}

\maketitle

\begin{abstract}
  The origins of combinatorial design theory lie within recreational mathematics. With the work of Fisher and Yates in the 1930s, combinatorial design theory began to take on the character of a serious academic subject with its applications to statistical experimentation, tournament scheduling, mathematical biology, algorithm design and analysis, and cryptography.\cite{Stinson:2008:CDC:1466390.1466393} This paper aims to provide an introduction to BIBDs (Balanced Incomplete Block Designs), identify the necessary conditions on $v$, $k$, and $\lambda$ such that (v, k, $\lambda$)-BIBDs exist, and discuss the application of BIBDs to the Social Golfer Problem.
\end{abstract}


\section{Introduction}
The statistician F. Yates studied subsets of a set subject to certain balance properties in his 1936 paper. \cite{Yates} In the paper he defined what has become known as \textit{(v, k, $\lambda$) balanced incomplete block designs}: 
\begin{mydef}\label{bibd}
  A \textit{$\left\{v, k, \lambda \right\}$ balanced incomplete block design} (BIBD) is a collection of $k$-element 
  subsets (called $blocks$) of a $v$-element set $S$ ($k$ $<$ v) such that each 2-element 
  subset of $S$ is contained in exactly $\lambda$ blocks.\cite{Yates}
\end{mydef}
A BIBD is called ``balanced'' because every pair of distinct elements is contained in exactly $\lambda$ blocks. Since $k$ $<$ $v$, no block can contain all elements of $S$, hence the designation ``incomplete''. Yates gave the following example of a (6,3,2)-BIBD in his paper: 
\begin{ex}
  S = $\left\{a, b, c, d, e, f\right\}$ with the following 10 3-element blocks:  
$\left\{a, b, c\right\}$, $\left\{a, b, d\right\}$, $\left\{a, c, e\right\}$, $\left\{a, d, f\right\}$, $\left\{a, e, f\right\}$, $\left\{b, c, e\right\}$, $\left\{b, d, e\right\}$, $\left\{b, e, f\right\}$, $\left\{c, d, e\right\}$, and $\left\{c, d, f\right\}$. 
\end{ex}
In this example, every element occurs in the same number of blocks. This property actually holds for all BIBDs, as shown in the following proof. \cite{Colbourn} 

\begin{proof}\label{pf1}
Let $x$ be an arbitrary element such that $x \in S$. Suppose $x$ occurs in $r$ blocks. In each of these blocks, $x$ makes a pair with $k-1$ other elements, so altogether there are $r(k-1)$ pairs involving $x$. But, $x$ is paired with the other $v-1$ elements exactly $\lambda$ times, so $r(k-1) = \lambda(v-1)$. This shows that $r$ is independent of the choice of $x$, being uniquely determined by $v$, $k$, and $\lambda$. 
\end{proof}
Thus, in a ($v, k, \lambda$) design with $b$ blocks, each element occurs in exactly $r$ blocks, and, as noted in the \textit{CRC Handbook of Combinatorial Designs}\cite{Colbourn}, the following equivalencies hold:
\begin{center}
\begin{itemize}
\item $\lambda(v-1) = r(k-1)$
\item $bk = vr$
\end{itemize}
\end{center}

The first equivalence was established in the first proof. The second equivalence can be explained by noting that since each of the $v$ elements appear in $r$ blocks, there are $vr$ appearances of elements in blocks. And since each of the $b$ blocks has $k$ elements, $vr = bk$.

Fisher also proved that the following inequality must hold: 
\begin{itemize}
\item $b \geq $v
\end{itemize}
When the equality holds, any two blocks have the same number of elements in common. (//TODO: look into this more later)
\begin{proof}
Let A be the block incidence matrix of a ($v$, $k$, $\lambda$)-BIBD with $b$ blocks. The block incidence matrix is a $v$ x $b$ incidence matrix A = ($a_{ij}$) such that:  
\begin{equation*}
a_{ij} = \begin{cases}
               1 & : \text{ if the } i^{th} \text { element is in the } j^{th} \text { block} \\
               0 & : \text{ otherwise}
           \end{cases}
\end{equation*}
\begin{center}
(See Example \ref{incidence-matrix})
\end{center}
If: 
\begin{itemize}
\item $A$ is $v\times{b}$
\item $A^T$ is the transpose\footnote{A matrix formed by turning all rows into columns and vice-versa} of $A$
\item $J_v$ is the all $v\times{v}$ 1's matrix
\item $I_v$ is the $v\times{v}$ identity matrix\footnote{A square matrix with 1`s on the main diagonal and 0's elsewhere}
\end{itemize}
Then: 
\begin{equation}
A^T \cdot A = \left({a_{ij}}\right) = \left({r - \lambda}\right) I_v + \lambda J_v
\end{equation} 
That is:
\begin{equation}
A^T \cdot A = \begin{bmatrix} 
r & \lambda & \cdots & \lambda \\
\lambda & r & \cdots & \lambda \\
\vdots & \vdots & \ddots & \vdots \\
\lambda & \lambda & \cdots & r \\
\end{bmatrix}
\end{equation}
From Necessary Condition for existence of BIBD\cite{necessaryBIBDCondition}:
\begin{equation*}
r > \lambda
\end{equation*}
So we can write $r = \lambda + \mu$ for some $\mu>0$ and so:
\begin{equation}
A^\intercal \cdot A = \begin{bmatrix} 
\lambda + \mu & \lambda & \cdots & \lambda \\
\lambda & \lambda + \mu & \cdots & \lambda \\
\vdots & \vdots & \ddots & \vdots \\
\lambda & \lambda & \cdots & \lambda + \mu \\
\end{bmatrix}
\end{equation}
This is a combinatorial matrix of order $v$.
So: 
\begin{equation}
\displaystyle \det \left({A^\intercal \cdot A}\right) = \displaystyle \mu^{v-1} \left ({\mu + v \lambda}\right)
\end{equation}
\begin{equation}
\displaystyle \det \left({A^\intercal \cdot A}\right) = \displaystyle \left ({r + \left({v-1}\right) \lambda}\right) \left({r - \lambda}\right)^{v-1}
\end{equation}
\begin{equation}
\displaystyle \det \left({A^\intercal \cdot A}\right) = \displaystyle r k \left({r - \lambda}\right)^{v-1}
\end{equation}
Now since $k < v$ and using the properties of a BIBD, we get that $r > \lambda$. So $\det \left({A^\intercal A}\right) \ne 0$. Since $A^\intercal A$ is a $v\times{v}$ matrix, then the rank, $\rho$, of $A^\intercal A = v$.
Using the facts that $\rho \left({A^\intercal A}\right) \le \rho \left({A}\right)$ and $\rho \left({A}\right) \le b$, (since $A$ only has $b$ columns), then $v \le \rho \left({A}\right) \le b$. \cite{FisherInequality}
\end{proof}
\begin{ex}\label{incidence-matrix}
The incidence matrix, A, of a (7, 3, 1)-BIBD consisting of 7 blocks (numbered 1, 2, 3, 4, 5, 6, and 7 respectively):
\begin{center}
\begin{tabular}{lllllll} 
(1,2,3) & (1,4,5) & (1,6,7) & (2,4,6) & (2,5,7) &  (3,4,7) & (3,5,6) \\
\end{tabular}
\end{center}
\[A = \left( \begin{array}{ccccccc}
1 & 1 & 1 & 0 & 0 & 0 & 0 \\
1 & 0 & 0 & 1 & 1 & 0 & 0 \\
1 & 0 & 0 & 0 & 0 & 1 & 1 \\
0 & 1 & 0 & 1 & 0 & 1 & 0 \\
0 & 1 & 0 & 0 & 1 & 0 & 1 \\
0 & 0 & 1 & 1 & 0 & 0 & 1 \\
0 & 0 & 1 & 0 & 1 & 1 & 0 \end{array} \right)\]
\end{ex}
\section{Resolvable Designs}
\begin{mydef}
A BIBD is \textit{resolvable} if its blocks can be partitioned into $c$ classes such that each element of the design occurs in exactly one of the $v/k = b/c$ groups of each class. The classes are called \textit{parallel classes} or \textit{resolution classes}. The partition into classess is called a \textit{resolution}. \cite{Triska}
\end{mydef}
Consider the following (15, 3, 1)-design consisting of 35 blocks:
\begin{center}
\begin{tabular}{lllllll} 
(1,2,3) & (1,4,7 ) & (1,5,10) & (1,6,15) & (1,8,11) &  (1,9,13) & (1,12,14) \\
(4,5,6) & (2,5,8) & (2,6,12) & (2,4,13) &  (2,7,14) & (2,11,15) & (2,9,10) \\
(7,8,9) & (3,10,13) & (3,8,15) & (3,7,12) &  (3,6,9) & (3,5,14) & (3,4,11) \\
(10,11,12) & (6,11,14) & (4,9,14) & (5,9,11) &  (4,10,15) & (4,8,12) & (5,7,15) \\
(13,14,15) & (9,12,15) & (7,11,13) & (8,10,14) &  (5,12,13) & (6,7,10) & (6,8,13)
\end{tabular}
\end{center}

\begin{center}
Each vertical group of 5 blocks is a resolution class.
\end{center}
\begin{thm}
If $D$ is a resolvable ($v$, $k$, 1) design, then a resolution of $D$ into $w$ parallel classes is a solution for the SGP instance $\frac{v}{k}$--$k$--$w$. \cite{Triska}
\end{thm}
The $v$ golfers correspond to elements $\left\{1, 2, 3, ...,v\right\}$. The design's blocks correspond to the group of golfers and the $w$ parallel classes are regarded as the weeks. By definition, each 2-element subset of $\left\{1, ...,v\right\}$ is contained in exactly one block. \cite{Triska}

\subsection{Necessary and Sufficient Conditions for Resolvability}
It has been shown in \cite{Bose} that for a \textit{resolvable} balanced incomplete block design, Fisher's inequality can be replaced by the more stringent inequality: 
\begin{equation}
v \geq v + r - 1
\end{equation}
When the equality holds, then two blocks belonging to different classes have the same number of elements in common. Balanced incomplete block designs with this property may be called \textit{affine} resolvable. 
For a resolvable balanced incomplete block design, the number of elements common to two blocks of different classes has to be $k^{2}/v$. (This number must be \textit{integral})
\begin{proof}
Consider a bibd with parameters $v$, $b$, $r$, $k$, $\lambda$. If it is resolvable then $v = nk$ and $b = nr$ for some integer $n$ (recap of previously stated definition). The $b$ blocks are divisible into $r$ classes or $n$ blocks each such that the blocks of a given class give a complete replication (i.e. all the elements occur exactly once among the blocks of a given class). Let the blocks belonging to the ith class ($C_i$) be 
\begin{equation}
B_{i,1}, B_{i,2}, B_{i,3}, ... B_{i, n} (i = 0, 1, ... r-1)
\end{equation}
Let us take any particular block $B_{0,1}$ of class $C_0$ and let $l_{i, j}$ be the number of elements common to the block $B_{0,1}$ and the block $B_{i,j}$ of the class $C_i$, for i = 1, 2, ..., r - 1, j = 1, 2, ..., n. Let $m$ denote the mean.
Each of the $k$ elements occuring in $B_{0,1}$ is replicated $r$ times. If an element occurs in $B_{0,1}$ then it cannot occur in the other blocks of the class $C_0$ (by definition of resolvability). Hence, it occurs just $r-1$ times among the blocks of the classes 
\end{proof}
\section{Steiner Triple Systems}
\begin{mydef}
A \textit{Steiner triple system} STS($v$) of order $v$ is a ($v$, 3, 1)-BIBD.
\end{mydef}

\clearpage


\bibliography{mybib}{}
\bibliographystyle{plain}


\end{document}